\documentclass[chaptersright]{informeutn}
\usepackage{circuitikz}
\usepackage{karnaugh-map}

\materia{Tecnicas Digitales }
\titulo{Trabajo Práctico 1}
\comision{3R2}
\autores{
          Gaston Grasso - 401892\\
          Franco Palombo - 401910\\
      }
\fecha{2/06/2025}

\begin{document}
  \maketitle

  \tableofcontents
  \setcounter{page}{1}
  \thispagestyle{plain}


\begin{tabular}{|c|c|c|c|c||c|c|c|c|}
\hline
\multicolumn{5}{|c||}{Entrada BCD} & \multicolumn{4}{c|}{Salida XS-3} \\
A & B & C & D & & S3 & S2 & S1 & S0 \\
\hline
0 & 0 & 0 & 0 & (0) & 0 & 0 & 1 & 1 \\
0 & 0 & 0 & 1 & (1) & 0 & 1 & 0 & 0 \\
0 & 0 & 1 & 0 & (2) & 0 & 1 & 0 & 1 \\
0 & 0 & 1 & 1 & (3) & 0 & 1 & 1 & 0 \\
0 & 1 & 0 & 0 & (4) & 0 & 1 & 1 & 1 \\
0 & 1 & 0 & 1 & (5) & 1 & 0 & 0 & 0 \\
0 & 1 & 1 & 0 & (6) & 1 & 0 & 0 & 1 \\
0 & 1 & 1 & 1 & (7) & 1 & 0 & 1 & 0 \\
1 & 0 & 0 & 0 & (8) & 1 & 0 & 1 & 1 \\
1 & 0 & 0 & 1 & (9) & 1 & 1 & 0 & 0 \\
\hline
\end{tabular}


\begin{karnaugh-map}[4][4][1][$CD$][$AB$]
    \minterms{5,6,7,8,9}
    \implicant{5}{7}
    \implicant{7}{6}
    \implicant{8}{9}
\end{karnaugh-map}

\begin{karnaugh-map}[4][4][1][$CD$][$AB$]
    \minterms{1,2,3,4}
    \implicant{1}{3}
    \implicant{3}{2}
\end{karnaugh-map}

\begin{karnaugh-map}[4][4][1][$CD$][$AB$]
    \minterms{0,3,4,7,8}
    \implicant{0}{4}
    \implicant{3}{7}
    \implicantcorner{8}{0}
\end{karnaugh-map}

\begin{karnaugh-map}[4][4][1][$CD$][$AB$]
    \minterms{0,2,4,6,8}
\end{karnaugh-map}

\end{document}

