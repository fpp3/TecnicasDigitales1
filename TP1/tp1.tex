\documentclass[chaptersright]{informeutn}
\usepackage{circuitikz}
\usepackage[label=corner]{karnaugh-map}

\materia{Tecnicas Digitales }
\titulo{Trabajo Práctico 1}
\comision{3R2}
\autores{
          Gaston Grasso - 401892\\
          Franco Palombo - 401910\\
      }
\fecha{2/06/2025}

\begin{document}
  \maketitle

  \tableofcontents
  \setcounter{page}{1}
  \thispagestyle{plain}


\newpage
\centering
\begin{tabular}{|c|c|c|c|c||c|c|c|c|}
\hline
\multicolumn{5}{|c||}{Entrada BCD} & \multicolumn{4}{c|}{Salida XS-3} \\
A & B & C & D & & S3 & S2 & S1 & S0 \\
\hline
0 & 0 & 0 & 0 & (0) & 0 & 0 & 1 & 1 \\
0 & 0 & 0 & 1 & (1) & 0 & 1 & 0 & 0 \\
0 & 0 & 1 & 0 & (2) & 0 & 1 & 0 & 1 \\
0 & 0 & 1 & 1 & (3) & 0 & 1 & 1 & 0 \\
0 & 1 & 0 & 0 & (4) & 0 & 1 & 1 & 1 \\
0 & 1 & 0 & 1 & (5) & 1 & 0 & 0 & 0 \\
0 & 1 & 1 & 0 & (6) & 1 & 0 & 0 & 1 \\
0 & 1 & 1 & 1 & (7) & 1 & 0 & 1 & 0 \\
1 & 0 & 0 & 0 & (8) & 1 & 0 & 1 & 1 \\
1 & 0 & 0 & 1 & (9) & 1 & 1 & 0 & 0 \\
\hline
\end{tabular}

\begin{minipage}[t]{0.48\textwidth}
    \centering
    \begin{karnaugh-map}[4][4][1][$CD$][$AB$]
        \minterms{5,6,7,8,9}
        \maxterms{0,1,2,3,4}
        \autoterms[X]
        \implicant{5}{15}
        \implicant{7}{14}
        \implicant{12}{10}
    \end{karnaugh-map}
    \captionof{figure}{Mapa de Karnaught para S3}
\end{minipage}
\hfill
\begin{minipage}[t]{0.48\textwidth}
    \centering
    $S3=A+B\cdot(C+D)$
\end{minipage}

\vspace{1cm}

\noindent
\begin{minipage}[t]{0.48\textwidth}
    \centering
    \begin{karnaugh-map}[4][4][1][$CD$][$AB$]
        \minterms{1,2,3,4,9}
        \maxterms{0,5,6,7,8}
        \autoterms[X]
        \implicantedge{1}{3}{9}{11}
        \implicantedge{3}{2}{11}{10}
        \implicant{4}{12}
    \end{karnaugh-map}
    \captionof{figure}{Mapa de Karnaught para S2}
\end{minipage}
\hfill
\begin{minipage}[t]{0.48\textwidth}
    \centering
    $S2=\overline{B}\cdot (C +  D) + B \cdot \overline{C} \cdot \overline{D}$
\end{minipage}

\noindent
\begin{minipage}[t]{0.48\textwidth}
    \centering
    \begin{karnaugh-map}[4][4][1][$CD$][$AB$]
        \minterms{0,3,4,7,8}
        \maxterms{1,2,5,6,9}
        \autoterms[X]
        \implicant{3}{11}
        \implicant{0}{8}
    \end{karnaugh-map}
    \captionof{figure}{Mapa de Karnaught para S1}
\end{minipage}
\hfill
\begin{minipage}[t]{0.48\textwidth}
    \centering
    $S1=C \cdot D + \overline{C} \cdot \overline{D}$
\end{minipage}

\noindent
\begin{minipage}[t]{0.48\textwidth}
    \centering
    \begin{karnaugh-map}[4][4][1][$CD$][$AB$]
        \minterms{0,2,4,6,8}    
        \maxterms{1,3,5,7,9}
        \autoterms[X]
        \implicantedge{0}{8}{2}{10}
    \end{karnaugh-map}
    \captionof{figure}{Mapa de Karnaught para S0}
\end{minipage}
\hfill
\begin{minipage}[t]{0.48\textwidth}
    \centering
    $S0=\overline{D}$
\end{minipage}




\end{document}

