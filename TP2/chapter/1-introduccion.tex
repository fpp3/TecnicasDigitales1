\chapter{Introducción}
El presente trabajo práctico tiene como objetivo estudiar e implementar un decodificador BCD a 7 segmentos utilizando
diferentes enfoques. En primer lugar, se realizó el armado del circuito integrado CD4511 en protoboard, lo que permitió
comprobar en la práctica el funcionamiento de sus entradas y salidas, así como el rol de las resistencias limitadoras de
corriente para proteger tanto al display como al integrado. Posteriormente, se implementó el mismo decodificador en
lenguaje Verilog, verificando su comportamiento mediante simulaciones antes de llevarlo a la placa de desarrollo CPLD. De
esta manera, se buscó integrar la teoría con la práctica, aplicando conceptos de técnicas digitales tanto en hardware
como en diseño asistido por computadora.
