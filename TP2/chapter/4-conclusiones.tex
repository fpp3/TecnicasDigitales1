\chapter{Conclusiones}
A lo largo del desarrollo de este trabajo se comprobó el correcto funcionamiento del decodificador BCD a 7 segmentos,
tanto en su implementación física con el CD4511 como en su versión programada en Verilog. La práctica permitió afianzar
conocimientos sobre el manejo de codificaciones binarias y su conversión a representaciones visuales en displays, así
como la importancia del uso de resistencias para la protección de los componentes. Además, la simulación en Verilog
demostró ser una herramienta útil para validar el diseño antes de su implementación en hardware. En conclusión, la
actividad facilitó la comprensión integral del tema y reforzó la relación entre la teoría estudiada y la práctica en
laboratorio.
